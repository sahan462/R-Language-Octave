% Options for packages loaded elsewhere
\PassOptionsToPackage{unicode}{hyperref}
\PassOptionsToPackage{hyphens}{url}
%
\documentclass[
]{article}
\usepackage{amsmath,amssymb}
\usepackage{iftex}
\ifPDFTeX
  \usepackage[T1]{fontenc}
  \usepackage[utf8]{inputenc}
  \usepackage{textcomp} % provide euro and other symbols
\else % if luatex or xetex
  \usepackage{unicode-math} % this also loads fontspec
  \defaultfontfeatures{Scale=MatchLowercase}
  \defaultfontfeatures[\rmfamily]{Ligatures=TeX,Scale=1}
\fi
\usepackage{lmodern}
\ifPDFTeX\else
  % xetex/luatex font selection
\fi
% Use upquote if available, for straight quotes in verbatim environments
\IfFileExists{upquote.sty}{\usepackage{upquote}}{}
\IfFileExists{microtype.sty}{% use microtype if available
  \usepackage[]{microtype}
  \UseMicrotypeSet[protrusion]{basicmath} % disable protrusion for tt fonts
}{}
\makeatletter
\@ifundefined{KOMAClassName}{% if non-KOMA class
  \IfFileExists{parskip.sty}{%
    \usepackage{parskip}
  }{% else
    \setlength{\parindent}{0pt}
    \setlength{\parskip}{6pt plus 2pt minus 1pt}}
}{% if KOMA class
  \KOMAoptions{parskip=half}}
\makeatother
\usepackage{xcolor}
\usepackage[margin=1in]{geometry}
\usepackage{color}
\usepackage{fancyvrb}
\newcommand{\VerbBar}{|}
\newcommand{\VERB}{\Verb[commandchars=\\\{\}]}
\DefineVerbatimEnvironment{Highlighting}{Verbatim}{commandchars=\\\{\}}
% Add ',fontsize=\small' for more characters per line
\usepackage{framed}
\definecolor{shadecolor}{RGB}{248,248,248}
\newenvironment{Shaded}{\begin{snugshade}}{\end{snugshade}}
\newcommand{\AlertTok}[1]{\textcolor[rgb]{0.94,0.16,0.16}{#1}}
\newcommand{\AnnotationTok}[1]{\textcolor[rgb]{0.56,0.35,0.01}{\textbf{\textit{#1}}}}
\newcommand{\AttributeTok}[1]{\textcolor[rgb]{0.13,0.29,0.53}{#1}}
\newcommand{\BaseNTok}[1]{\textcolor[rgb]{0.00,0.00,0.81}{#1}}
\newcommand{\BuiltInTok}[1]{#1}
\newcommand{\CharTok}[1]{\textcolor[rgb]{0.31,0.60,0.02}{#1}}
\newcommand{\CommentTok}[1]{\textcolor[rgb]{0.56,0.35,0.01}{\textit{#1}}}
\newcommand{\CommentVarTok}[1]{\textcolor[rgb]{0.56,0.35,0.01}{\textbf{\textit{#1}}}}
\newcommand{\ConstantTok}[1]{\textcolor[rgb]{0.56,0.35,0.01}{#1}}
\newcommand{\ControlFlowTok}[1]{\textcolor[rgb]{0.13,0.29,0.53}{\textbf{#1}}}
\newcommand{\DataTypeTok}[1]{\textcolor[rgb]{0.13,0.29,0.53}{#1}}
\newcommand{\DecValTok}[1]{\textcolor[rgb]{0.00,0.00,0.81}{#1}}
\newcommand{\DocumentationTok}[1]{\textcolor[rgb]{0.56,0.35,0.01}{\textbf{\textit{#1}}}}
\newcommand{\ErrorTok}[1]{\textcolor[rgb]{0.64,0.00,0.00}{\textbf{#1}}}
\newcommand{\ExtensionTok}[1]{#1}
\newcommand{\FloatTok}[1]{\textcolor[rgb]{0.00,0.00,0.81}{#1}}
\newcommand{\FunctionTok}[1]{\textcolor[rgb]{0.13,0.29,0.53}{\textbf{#1}}}
\newcommand{\ImportTok}[1]{#1}
\newcommand{\InformationTok}[1]{\textcolor[rgb]{0.56,0.35,0.01}{\textbf{\textit{#1}}}}
\newcommand{\KeywordTok}[1]{\textcolor[rgb]{0.13,0.29,0.53}{\textbf{#1}}}
\newcommand{\NormalTok}[1]{#1}
\newcommand{\OperatorTok}[1]{\textcolor[rgb]{0.81,0.36,0.00}{\textbf{#1}}}
\newcommand{\OtherTok}[1]{\textcolor[rgb]{0.56,0.35,0.01}{#1}}
\newcommand{\PreprocessorTok}[1]{\textcolor[rgb]{0.56,0.35,0.01}{\textit{#1}}}
\newcommand{\RegionMarkerTok}[1]{#1}
\newcommand{\SpecialCharTok}[1]{\textcolor[rgb]{0.81,0.36,0.00}{\textbf{#1}}}
\newcommand{\SpecialStringTok}[1]{\textcolor[rgb]{0.31,0.60,0.02}{#1}}
\newcommand{\StringTok}[1]{\textcolor[rgb]{0.31,0.60,0.02}{#1}}
\newcommand{\VariableTok}[1]{\textcolor[rgb]{0.00,0.00,0.00}{#1}}
\newcommand{\VerbatimStringTok}[1]{\textcolor[rgb]{0.31,0.60,0.02}{#1}}
\newcommand{\WarningTok}[1]{\textcolor[rgb]{0.56,0.35,0.01}{\textbf{\textit{#1}}}}
\usepackage{graphicx}
\makeatletter
\def\maxwidth{\ifdim\Gin@nat@width>\linewidth\linewidth\else\Gin@nat@width\fi}
\def\maxheight{\ifdim\Gin@nat@height>\textheight\textheight\else\Gin@nat@height\fi}
\makeatother
% Scale images if necessary, so that they will not overflow the page
% margins by default, and it is still possible to overwrite the defaults
% using explicit options in \includegraphics[width, height, ...]{}
\setkeys{Gin}{width=\maxwidth,height=\maxheight,keepaspectratio}
% Set default figure placement to htbp
\makeatletter
\def\fps@figure{htbp}
\makeatother
\setlength{\emergencystretch}{3em} % prevent overfull lines
\providecommand{\tightlist}{%
  \setlength{\itemsep}{0pt}\setlength{\parskip}{0pt}}
\setcounter{secnumdepth}{-\maxdimen} % remove section numbering
\ifLuaTeX
  \usepackage{selnolig}  % disable illegal ligatures
\fi
\IfFileExists{bookmark.sty}{\usepackage{bookmark}}{\usepackage{hyperref}}
\IfFileExists{xurl.sty}{\usepackage{xurl}}{} % add URL line breaks if available
\urlstyle{same}
\hypersetup{
  pdftitle={q2.R},
  pdfauthor={ASUS},
  hidelinks,
  pdfcreator={LaTeX via pandoc}}

\title{q2.R}
\author{ASUS}
\date{2023-12-04}

\begin{document}
\maketitle

\begin{Shaded}
\begin{Highlighting}[]
\CommentTok{\# Create the dataframe}
\NormalTok{emp\_sal }\OtherTok{\textless{}{-}} \FunctionTok{data.frame}\NormalTok{(}
  \AttributeTok{Emp\_ID =} \FunctionTok{c}\NormalTok{(}\DecValTok{11}\NormalTok{, }\DecValTok{12}\NormalTok{, }\DecValTok{13}\NormalTok{, }\DecValTok{14}\NormalTok{, }\DecValTok{15}\NormalTok{),}
  \AttributeTok{Dep =} \FunctionTok{c}\NormalTok{(}\StringTok{"Sales"}\NormalTok{, }\StringTok{"HR"}\NormalTok{, }\StringTok{"Sales"}\NormalTok{, }\StringTok{"HR"}\NormalTok{, }\StringTok{"Sales"}\NormalTok{),}
  \AttributeTok{Basic =} \FunctionTok{c}\NormalTok{(}\DecValTok{25450}\NormalTok{, }\DecValTok{22500}\NormalTok{, }\DecValTok{21000}\NormalTok{, }\DecValTok{23500}\NormalTok{, }\DecValTok{15000}\NormalTok{),}
  \AttributeTok{Allowances =} \FunctionTok{c}\NormalTok{(}\DecValTok{5200}\NormalTok{, }\DecValTok{4500}\NormalTok{, }\DecValTok{3100}\NormalTok{, }\DecValTok{2600}\NormalTok{, }\DecValTok{1800}\NormalTok{)}
\NormalTok{)}

\FunctionTok{print}\NormalTok{(emp\_sal)}
\end{Highlighting}
\end{Shaded}

\begin{verbatim}
##   Emp_ID   Dep Basic Allowances
## 1     11 Sales 25450       5200
## 2     12    HR 22500       4500
## 3     13 Sales 21000       3100
## 4     14    HR 23500       2600
## 5     15 Sales 15000       1800
\end{verbatim}

\begin{Shaded}
\begin{Highlighting}[]
\CommentTok{\#Store net salary in new column named “net\_sal”}
\NormalTok{emp\_sal}\SpecialCharTok{$}\NormalTok{net\_salary }\OtherTok{\textless{}{-}}\NormalTok{ emp\_sal}\SpecialCharTok{$}\NormalTok{Basic }\SpecialCharTok{+}\NormalTok{ emp\_sal}\SpecialCharTok{$}\NormalTok{Allowances}
\FunctionTok{print}\NormalTok{(emp\_sal}\SpecialCharTok{$}\NormalTok{net\_salary)}
\end{Highlighting}
\end{Shaded}

\begin{verbatim}
## [1] 30650 27000 24100 26100 16800
\end{verbatim}

\begin{Shaded}
\begin{Highlighting}[]
\CommentTok{\#Obtain employee IDs of employees whose net salary is above 25000}
\NormalTok{high\_salary\_employees }\OtherTok{\textless{}{-}}\NormalTok{ emp\_sal}\SpecialCharTok{$}\NormalTok{Emp\_ID[emp\_sal}\SpecialCharTok{$}\NormalTok{net\_sal }\SpecialCharTok{\textgreater{}} \DecValTok{25000}\NormalTok{]}
\FunctionTok{print}\NormalTok{(high\_salary\_employees)}
\end{Highlighting}
\end{Shaded}

\begin{verbatim}
## [1] 11 12 14
\end{verbatim}

\begin{Shaded}
\begin{Highlighting}[]
\CommentTok{\#Obtain employee IDs of employees attached to HR Department whose net salary is below 25000}
\NormalTok{HR\_high\_salary\_employees }\OtherTok{\textless{}{-}}\NormalTok{ emp\_sal}\SpecialCharTok{$}\NormalTok{Emp\_ID[emp\_sal}\SpecialCharTok{$}\NormalTok{net\_sal }\SpecialCharTok{\textgreater{}} \DecValTok{25000} \SpecialCharTok{\&}\NormalTok{ emp\_sal}\SpecialCharTok{$}\NormalTok{Dep }\SpecialCharTok{==} \StringTok{\textquotesingle{}HR\textquotesingle{}}\NormalTok{]}
\FunctionTok{print}\NormalTok{(HR\_high\_salary\_employees)}
\end{Highlighting}
\end{Shaded}

\begin{verbatim}
## [1] 12 14
\end{verbatim}

\end{document}
